\documentclass[a4paper,11pt]{article}

% rubber: paper a4

\usepackage{amsmath}
\usepackage{amsthm}
\usepackage{amssymb}
\usepackage{mathrsfs}
\usepackage{varioref}
\usepackage{comment}

\newcommand{\rmd}{\textnormal{d}}
\newcommand{\bbP}{\mathbb{P}}
\newcommand{\bbR}{\mathbb{R}}
\newcommand{\lv}{\lvert}
\newcommand{\rv}{\rvert}
\newcommand{\lV}{\lVert}
\newcommand{\rV}{\rVert}
\newcommand{\isom}{\cong}

\newtheorem{thm}{Theorem}[section]
\newtheorem{lemma}[thm]{Lemma}
\newtheorem{corollary}[thm]{Corollary}
\newtheorem{defn}[thm]{Definition}

\sloppy

\begin{document}

\title{Isoperimetric inequality for compact subsets of Euclidean space}
\author{John Bytheway}
\date{June 2006}
\maketitle

\tableofcontents

\setlength{\parindent}{0pt}
\setlength{\parskip}{1ex plus 0.5ex minus 0.2ex}

\section{Introduction}
The purpose of this article is to present a (relatively) elementary proof of
a general isoperimetric inequality for compact subsets of $\bbR^m$.  We shall
use elementary point-set topology (primarily compactness in $\bbR^m$), and
slightly less elementary measure-theory (e.g. Fubini's Theorem).  At one point
(to prove \vref{thm:openNeighbourhoods}) we are forced to invoke a powerful
and far from elementary theorem, but hopefully the result is intuitively
`obvious'.

An isoperimetric inequality applies in a situation where a class of objects
have a concept of both `volume' and `surface area' (or `area' and `perimeter'
--- hence the name `isoperimetric'), and states a result of the form ``If $A$
has a volume of at least $x$ then $A$ has a surface area of at least $y$''.
Frequently the expression for the minimum surface area of any given volume is
quite ugly, and it is much more convinient to rephrase the result in the form
``Amongst all objects of volume $x$, the object $A$ has minimal surface area''.

We are concerned in particular with compact subsets of $\bbR^m$, and we show
that for suitable definitions of volume and surface area, closed balls minimise
surface area for any fixed volume.

\section{Spaces of compact subsets}
We are concerned with compact subsets of $\bbR^m$, and we need
some preliminary results about the properties of such compact subsets.

\begin{defn}
Let $(X,d)$ be a metric space.  We define
\[
\kappa(X)=\{K\subseteq X:K\textrm{ is compact and }K\not=\emptyset\}
\]
to be the \emph{space of compact subsets} of $X$.

We also define
\begin{eqnarray*}
\delta &:& \kappa(X)\times\kappa(X)\to\bbR \\
&& (A,B)\mapsto \sup_{a\in A}\inf_{b\in B}d(a,b) \\
d &:& \kappa(X)\times\kappa(X)\to\bbR \\
&& (A,B)\mapsto \delta(A,B)+\delta(B,A)
\end{eqnarray*}
\end{defn}

\begin{lemma}
\label{thm:deltaAndInclusion}
$\delta(A,B)=0\iff A\subseteq B$
\end{lemma}

\begin{proof}
Assume $A\subseteq B$ and let $a\in A$.  Then
$\inf_{b\in B}d(a,b)\leq\inf_{b\in A}d(a,b)\leq d(a,a)=0$ and hence $\inf_{b\in
B}d(a,b)=0$ so $\delta(A,B)=0$.

Conversely, assume $\delta(A,B)=0$ and let $a\in A$.  Certainly
$\inf_{b\in B}d(a,b)=0$, but B is compact, so we deduce that $a\in B$.  Thus
$A\subseteq B$ as required.
\end{proof}

\begin{lemma}
\label{thm:spaceOfCompactSubsets}
For any metric space $(X,d)$, the function $d:\kappa(X)\times\kappa(X)\to\bbR$
constructed in this way is a
metric on $\kappa(X)$.  Furthermore, if $X$ is complete, then so is
$\kappa(X)$, and if $X$ is totally bounded, then so is $\kappa(X)$.
\end{lemma}

\begin{proof}
To show that $d$ is a metric we must prove:
\begin{eqnarray*}
&&d(A,B)=0 \iff A=B \\
&&d(A,B)=d(B,A) \\
&&d(A,B)+d(B,C)\geq d(A,C)
\end{eqnarray*}
%
The first follows at once from \ref{thm:deltaAndInclusion}, and the second from
the definition of $d$.

To deduce the triangle inequality for $d$ it suffices to prove the triangle
inequality for $\delta$, which we do as follows:

Let $a\in A$.  We know $\inf_{b\in B}d(a,b)\leq\delta(A,B)$, so (by compactness
of B) we can find $b_0\in B$ s.t.
$d(a,b_0)\leq\delta(A,B)$.  Now similarly we can
find $c_0\in C$ s.t. $d(b_0,c_0)\leq\delta(B,C)$.
By the triangle inequality on
$(X,d)$ we deduce that:
%
\begin{eqnarray*}
\inf_{c\in C}d(a,c)&\leq&d(a,c_0) \\
  &\leq& d(a,b_0)+d(b_0,c_0) \\
  &\leq& \delta(A,B)+\delta(B,C)
\end{eqnarray*}
%
Taking the supremum over $a\in A$ yields the triangle inequality for $\delta$,
as required.

Next we wish to show that if $X$ is totally bounded, then so is $\kappa(X)$.

Let $\epsilon>0$, and let $N=\{x_0,\ldots,x_n\}$ be an $\frac{\epsilon}{4}$-net
for $X$.  We wish to show that
$\bbP(N)\setminus\{\emptyset\}$ (the power set of $N$
excepting the empty set) is an
$\epsilon$-net for $\kappa(X)$.

Take any $A\in\kappa(X)$.  Let $B=\{x\in N:d(x,A)<\frac{\epsilon}{4}\}$.
Certainly $B\in\bbP(N)\setminus\{\emptyset\}$.  We
will show that $d(A,B)<\epsilon$, and thus deduce that $\bbP(N)$ is indeed an
$\epsilon$-net.

The fact that $N$ is an $\frac{\epsilon}{4}$-net, together with the definition
of $B$, quickly tells us:
%
\begin{eqnarray*}
\delta(A,B)
  &=& \sup_{a\in A}\inf_{b\in B}d(a,b) \\
  &\leq& \frac{\epsilon}{4}
\end{eqnarray*}
%
The definition of $B$ alone is sufficient to show that
$\delta(B,A)\leq\frac{\epsilon}{4}$.  Thus
$d(A,B)\leq\frac{\epsilon}{2}<\epsilon$ as required.

The last and most tricky part of the proof is to show that if $X$ is complete
then so is $\kappa(X)$.

Let $A$ be a Cauchy sequence in $\kappa(X)$.  So
\[
\lim_{n\to\infty}\sup_{m>n}d(A_n,A_m)=0
\]
Choose a subsequence $B$ which
converges sufficiently fast that $\sup_{m>n}d(B_n,B_m)<2^{-n}$.  In particular
this means that
%
\begin{eqnarray*}
&&\sup_{m>n}\delta(B_n,B_m)<2^{-n}\textrm{ and} \\
&&\sup_{m>n}\delta(B_m,B_n)<2^{-n}
\end{eqnarray*}

Let $C$ be the set of all Cauchy sequences $x$ in $X$ satisfying
$\forall n:x_n\in B_n$, and let $L$ be the set of limits of these Cauchy
sequences.  We wish to show that $B_n\to L$ as $n\to\infty$.

First we need to show that $L\in\kappa(X)$.  This requires proving that $L$ is
compact and non-empty.

We show that $L$ is sequentially compact.  Let $y$ be a sequence in $L$ with
$k^{th}$ element $y^{(k)}$.  Each
$y^{(k)}$ is the limit of a sequence $x^{(k)}$ in $C$.  We take subsequences
infinitely many times as follows:

At the $i^{th}$ occassion of taking a subsequence, we will alter only the
elements from the $i^{th}$ onwards.  This ensures that we do have a sequence
left after these infinitely many operations.

At the $i^{th}$ occassion we start with a sequence of Cauchy sequences $x$ with
$x^{(k)}_n\to y^{(k)}$ as $n\to\infty$.  Also, we know that $x^{(k)}_n\in B_n$.
%
\begin{eqnarray*}
d(y^{(k)},B_i)
  &=& \lim_{n\to\infty}d(x^{(k)}_n,B_i) \\
  &\leq& \limsup_{n\to\infty}\delta(B_n,B_i) \\
  &\leq& 2^{-i}
\end{eqnarray*}
%
So, for each $k$, we can choose a $z_i^{(k)}\in B_i$ with
$d(y^{(k)},z_i^{(k)})\leq2^{-i}$.  Since $B_i$ is sequentially compact, the
sequence $z_i$ formed from these $z_i^{(k)}$
(as $k$ varies) has a convergent subsequence.  Choose
such a subsequence, but only discarding elements from the $i^{th}$ onwards, and
furthermore ensuring that, in this subsequence, when $k\geq i$ we have
$d(z_i^{(i)},z_i^{(k)})<2^{-i}$.  
Choose corresponding subsequences of $x$ and $y$ in preperation for the next
step.  Note that when $k\geq i$ we have
%
\begin{eqnarray*}
d(y^{(i)},y^{(k)})
  &\leq& d(y^{(i)},z_i^{(i)})+d(z_i^{(i)},z_i^{(k)})+d(z_i^{(k)},y^{(k)}) \\
  &<& 3\cdot2^{-i}
\end{eqnarray*}
%
We are left with a sequence $y$ for which the above equation holds for all $i$
and $k\geq i$.  Thus $y$ is Cauchy, and converges to
a limit $l$ (since $X$ is complete).  Furthermore,
$\forall k:d(y^{(k)},l)<3\cdot 2^{-i}$.  It
remains to be shown that $l\in L$.  We can deduce this by considering the
sequence $z_k^{(k)}$ (as $k$ varies), and noting that
%
\begin{eqnarray*}
d(z_k^{(k)},l)
  &\leq& d(z_k^{(k)},y^{(k)})+d(y^{(k)},l) \\
  &<& 4\cdot2^{-k}
\end{eqnarray*}
%
So, $z_k^{(k)}\to l$ as $k\to\infty$ and the sequence of $z_k^{(k)}$ is in $C$.
Thus indeed $l\in L$, and $L$ is sequentially compact.

To show that $L\not=\emptyset$ we must construct an
element $x$ of $C$.  We can do this
inductively as follows.  The first element $x_0\in B_0$ is chosen arbitrarily.
To choose an element $x_{n+1}$ we observe that $\delta(B_n,B_{n+1})<2^{-n}$, so
we can find $x_{n+1}\in B_{n+1}$ with $d(x_n,x_{n+1})<2^{-n}$.  This choice
clearly yields a Cauchy sequence $x$, so $L$ is indeed nonempty.

Next, we must show that $B_n\to L$ as $n\to\infty$
in $(\kappa(X),d)$.  It suffices to show
that $\delta(B_n,L)\to0$ and $\delta(L,B_n)\to0$.

To bound $\delta(B_n,L)$ we construct a sequence in much the same way as we did
to show that $L\not=\emptyset$.  This sequence starts with arbitrary $x_n\in
B_n$ and then each subsequent $x_{k+1}\in B_{k+1}$ satisfies
$d(x_k,x_{k+1})<2^{-k}$.  Thus $x\in C$ and $x_k\to l\in L$ with
$d(x_n,l)<2^{1-n}$.

But $x_n$ was chosen arbitrarily from $B_n$, so $\delta(B_n,L)\leq2^{1-n}\to0$
as $n\to\infty$.

To bound $\delta(L,B_n)$ we take any $l\in L$.  Imagine the constant sequence
in $L$ of which every element is $l$.  Trivially this sequence tends to $l$,
and is a Cauchy sequence in $L$.  Applying the same process we used to show
that $L$ is compact, we can construct from this constant sequence another
sequence $z$ with $z_k\to l$ as $k\to\infty$ and $z_k\in B_k$ and
$d(z_k,l)\leq4\cdot2^{-k}$.  In particular, we obtain $z_n\in B_n$ with
$d(z_n,l)\leq2^{2-n}$.  Thus $\delta(L,B_n)\leq2^{2-n}\to0$ as $n\to\infty$.

So, indeed, $B_n\to L$ as $n\to\infty$.  By general metric space results, if a
Cauchy sequence has a convergent subsequence then it is itself convergent, so
$A_n\to L$ as $n\to\infty$.  Thus $\kappa(X)$ is complete, as required.
\end{proof}

We, of course, are primarily concerned with $\bbR^m$, and always take
the standard Euclidean metric thereon.

\begin{corollary}
$\kappa(\bbR^m)$ is complete.
\end{corollary}

\begin{defn}
For a metric space $(X,d)$, an element $x\in X$ and $R\in\bbR$, let
$B(x,R)=\{y\in X:d(x,y)\leq R\}$ be the \emph{closed ball}
of radius $R$ centred at $x$, and let
$B^o(x,R)=\{y\in X:d(x,y)<R\}$ be the \emph{open ball}
of radius $R$ centred at $x$.

We write $\kappa(\bbR^m,R)$ for $\kappa(B(0,R))$ (thinking of $0\in\bbR^m$).
\end{defn}

So, by \ref{thm:spaceOfCompactSubsets}, we deduce that $\kappa(\bbR^m,R)$ is
compact.

Now in order to state an isoperimetric inequality we need to define concepts
analagous to both volume and surface area.  For the volume of
$A\in\kappa(\bbR^m)$ we take simply the standard lebesgue measure of the set,
which we denote by $\mu(A)$.

The following lemma demonstrates how volume interacts with the topology on
$\kappa(\bbR^m)$:

\begin{lemma}
\label{thm:upperContinuityOfVolume}
Let $K_n$ be a convergent sequence in $\kappa(\bbR^m)$ with limit $K$.  Then
$\mu(K)\geq\limsup_{n\to\infty}\mu(K_n)$.
\end{lemma}

\begin{proof}
Taking a result from measure theory, we know that
\[
\mu(K)=\inf\{\mu(U):K\subseteq U,\ U \textrm{ open}\}
\]
So let $U$ be open and containing $K$.  Let
$\epsilon=\inf_{k\in K}\inf_{x\in\bbR^m\setminus U}d(k,x)$.  By the choice of
$U$, we know $\epsilon>0$.

Since $K_n\to K$, eventually $\delta(K_n,K)<\epsilon$, and so eventually
$K_n\subseteq U$.
Thus eventually $\mu(K_n)\leq\mu(U)$, and hence
$\mu(U)\geq\limsup_{n\to\infty}\mu(K_n)$.

Taking the infimum over all eligible sets $U$, we get the result.
\end{proof}

This suggests a further definition and associated corollary:

\begin{defn}
For $R,V\in\bbR$, we define
\[
\kappa(\bbR^m,R,V)=\{K\in\kappa(\bbR^m,R):\mu(K)\geq V\}
\]
\end{defn}

\begin{corollary}
\label{thm:volumeBoundedSpaces}
The set $\kappa(\bbR^m,R,V)$ is a closed subset of $\kappa(\bbR^m,R)$, and is
thus compact.
\end{corollary}

To aquire a concept of surface area, we make several definitions:

\begin{defn}
We define the (\emph{closed} and \emph{open} respectively)
\emph{epsilon-neighbourhoods} of $K\in\kappa(\bbR^m)$ to be
\begin{eqnarray*}
N_\epsilon(K)&=&\{x\in\bbR^m:d(x,K)\leq \epsilon\} \\
N_\epsilon^o(K)&=&\{x\in\bbR^m:d(x,K)<\epsilon\}
\end{eqnarray*}
We also define
\[
S_\epsilon(K)=\mu(N_\epsilon(K))
\]
and then take the surface area of $K$ to be
\[
S(K)=\limsup_{\epsilon\to0^+}\frac{S_\epsilon(K)-\mu(K)}{\epsilon}
\]
(Effectively the right derivative with respect to $\epsilon$ of $S_\epsilon(K)$
at $\epsilon=0$.)
\end{defn}

Note that in some cases (for example the Cantor set in $\bbR$) this definition
gives an infinite surface area.

\begin{lemma}
\label{thm:openNeighbourhoods}
For any $K\in\kappa(\bbR^m)$, and any $\epsilon>0$:
\[
S_\epsilon(K)=\mu(N_\epsilon^o(K))
\]
\end{lemma}

\begin{proof}
Let $\partial_\epsilon(K)=\{x\in\bbR^m:d(x,K)=\epsilon\}$.  This lemma is
equivalent to the assertion that $\mu(\partial_\epsilon(K))=0$.

To prove this fact we invoke a very hard to prove result from functional
analysis:

\begin{thm}
Suppose $A\subseteq\bbR^m$ be a measurable set, with indicator function $f$.

Define, for $\eta>0$:
\[
f_\eta:\bbR^m\to\bbR;x\mapsto\frac{\mu(A\cap B^o(x,\eta))}{\mu(B^o(x,\eta))}
\]
Then $f_\eta\to f$ a.e. as $\eta\to0^+$.
\end{thm}

We apply this result with $A=\partial_\epsilon(K)$.

Let $x\in A$.  By definition of $\partial_\epsilon(K)$, we can find $k\in K$
with $d(x,k)=\epsilon$.  Certainly $B^o(k,\epsilon)\cap A=\emptyset$.

Assume $\eta<\epsilon$.  Note that $A\cap B^o(x,\eta)$ is disjoint from
$B^o(k,\epsilon)\cap B^o(x,\eta)$.  This latter is the intersection of two
balls of radii $\epsilon$ and $\eta$, with centres seperated by $\epsilon$.
Computing the actual volume of this intersection is theoretically possible, but
would be extremely messy in practice, so I ask instead that the reader
mentally visualize what happens as $\eta\to0^+$, with $\epsilon$ fixed.
It should be clear that
\[
\frac{\mu(B^o(k,\epsilon)\cap B^o(x,\eta))}{\mu(B^o(x,\eta))}\to\frac{1}{2}
\]
Thus, by the previously observed disjointness
\[
\limsup_{\eta\to0^+}\frac{\mu(A\cap B^o(x,\eta))}{\mu(B^o(x,\eta))}
  \leq\frac{1}{2}
\]
Or, in the notation of the theorem:
\[
\limsup_{\eta\to0^+}f_\eta(x)\leq\frac{1}{2}
\]
but $f(x)=1$ (since $x\in A$ and $f$ is the indicator function of $A$).

In summary, $\forall x\in A:f_\eta(x)\not\to f(x)$ as $\eta\to0^+$.

But the theorem states that $f_\eta\to f$ a.e., so the only possible resolution
is that $\mu(A)=0$, as required.
\end{proof}

\begin{comment}
We should check that this new definition of surface area agress with any other
definition of surface area we might encounter in certain cases.  We so so in
the following theorem, which is not required for the remaining results, but
helps to ensure that our effort is worthwhile.

\begin{thm}
\label{thm:smoothSurfaceAreas}
When $K$ is a compact smooth submanifold (with boundary) of $\bbR^m$, $S(K)$
agrees with the standard definition of surface area in terms of the integral
over the boundary manifold of the volume form induced by the Riemannian metric
induced by the ambient metric on $\bbR^m$.
\end{thm}

\begin{proof}
TODO (not much of an idea yet).
\end{proof}
\end{comment}

Again we need to determine how this concept of surface area interacts with
the topology.

\begin{lemma}
\label{thm:surfaceAreaContinuous}
For all $\epsilon>0$, both $N_\epsilon$ and $S_\epsilon$ are continuous.
\end{lemma}

(Note that $S$ is very much not continuous, and that although $N_0$ is
continuous (it is the identity), $S_0$ is not (it is simply $\mu$))

\begin{proof}
Fix $\epsilon>0$.

Let $A\in\kappa(\bbR^m)$.  Let $\eta>0$.

Suppose $d(A,B)\leq\frac{\eta}{2}$.  Then certainly both
$\delta(A,B)\leq\frac{\eta}{2}$ and $\delta(B,A)\leq\frac{\eta}{2}$.

Let $c\in N_\epsilon(A)$.  Then we can deduce as follows:
%
\begin{eqnarray*}
\exists a\in A &:& d(a,c)\leq\epsilon \\
\exists b\in B &:& d(a,b)\leq\frac{\eta}{2} \\
&& d(b,c)\leq\epsilon+\frac{\eta}{2} \\
&& \textrm{By the triangle inequality} \\
\exists d\in N_\epsilon(B) &:& d(c,d)\leq\frac{\eta}{2} \\
&& \textrm{By taking $d$ on the straight line between $b$ and $c$} \\
&&\inf_{d\in N_\epsilon(B)}d(c,d)\leq\frac{\eta}{2} \\
&&\sup_{c\in N_\epsilon(A)}\inf_{d\in N_\epsilon(B)}d(c,d)\leq\frac{\eta}{2} \\
&&\delta(N_\epsilon(A),N_\epsilon(B))\leq\frac{\eta}{2}
\end{eqnarray*}
%
By symmetry, we also have
$\delta(N_\epsilon(B),N_\epsilon(A))\leq\frac{\eta}{2}$, and thus
$d(N_\epsilon(A),N_\epsilon(B))\leq\eta$, and hence $N_\epsilon$ is continuous.

Together with \ref{thm:upperContinuityOfVolume},
this is enough to tell us that for
$K_n\to K$, we have $S_\epsilon(K)\geq\limsup S_\epsilon(K_n)$, and so to get
continuity of $S_\epsilon$ we need to deduce that
\[
S_\epsilon(K)\leq\liminf_{n\to\infty} S_\epsilon(K_n)
\]
First, using \ref{thm:openNeighbourhoods}, we rephrase this as
\[
\mu(N_\epsilon^o(K))\leq\liminf_{n\to\infty}\mu(N_\epsilon(K_n))
\]
Now, taking another measure theoretic result, we know that the left hand side
of this inequality is
\[
\sup\{\mu(A):A\textrm{ is compact and }A\subseteq N_\epsilon^o(K)\}
\]
Let $A$ be such a set.  Note that
\[
N_\epsilon^o(K)=\bigcup_{\eta<\epsilon}N_\eta^o(K)
\]
so the $N_\eta^o(K)$ form an open cover of $A$, and we can take a finite
subconer which then has a maximal element (when ordered by inclusion).  Thus
$\exists\eta\in(0,\epsilon):A\subseteq N_\eta(K)$.

Now, eventually $\delta(K,K_n)\leq\epsilon-\eta$, and so eventually
$A\subseteq N_\epsilon(K_n)$, and hence eventually
$\mu(A)\leq\mu(N_\epsilon(K_n))$, from which we get
$\mu(A)\leq\liminf\mu(N_\epsilon(K_n))$.

Taking the supremum over all eligible $A$ gives the result.
\end{proof}

Before we leave this section, we make one final definition which we shall need
later:

\begin{defn}
For $K\in\kappa(\bbR^m)$, we take
\[
\lV K\rV=\sup_{k\in K}\lV k\rV
\]
\end{defn}

So, $\kappa(\bbR^m,R)=\{K\in\kappa(\bbR^m):\lV K\rV\leq R\}$.

%%%%%%%%%%%%%%%%%%%%%%%%%%%%%%%%%%%%%%%%%%%%%%%%%%%%%%%%%%
%%%%%%%%%%%%%%%%%%%%%%%%%%%%%%%%%%%%%%%%%%%%%%%%%%%%%%%%%%
%%%%%%%%%%%%%%%%%%%%%%%%%%%%%%%%%%%%%%%%%%%%%%%%%%%%%%%%%%
\section{Compressions}
Our strategy with regard to proving the main result is to use
\emph{compressions}.

Given any compact set $A$, we try to compress it in such a way that we achieve
three goals:
\begin{enumerate}
\item Keep the volume fixed.
\item Do not increase the surface area.
\item Make $A$ look ``more like a closed ball''
\end{enumerate}
If we can manage all of these simultaneously, then we can imagine that if we
keep compressing long enough then eventually we will have made $A$ into a
closed ball of the same volume as $A$, but with a smaller surface area, thus
showing that closed balls minimise surface area amongst compact sets of any
fixed volume.

In order to do this, we need to define an appropriate compression, as follows:

\begin{defn}
\label{defn:compression}
Let $K\in\kappa(\bbR^m)$ and $v\in S^{m-1}$.

Let $P_v=v^\perp$ be the subspace orthogonal to $v$.

We define the \emph{$v$-sections} of any subset $A\subseteq\bbR^m$ by,
for each $x\in\bbR$
\[
A(v,x)=\{y\in P_v:y+xv\in A\}
\]
Now we define the \emph{$v$-compression} of $K$ to be $C_v(K)$, where,
$\forall x\in\bbR$:
%
\begin{eqnarray*}
\mu(C_v(K)(v,x))&=&\mu(K(v,x))
\end{eqnarray*}
%
(Where measures are calculated in $P_v$)
and $C_v(K)(v,x)$ is a closed ball centred at $0$ in $P_v$.

If $\mu(K(v,x))>0$, this defines $C(v,x)$ uniquely, but if $\mu(K(v,x))=0$,
then we could take $C_v(K)(v,x)=\emptyset$, or $\{0\}$, since both of these are
closed balls of measure zero.  We use the latter iff, for every neighbourhood
$U$ of $x$ in $\bbR$, we can find $y\in U:\mu(K(v,y))>0$.

As a particular special case, if $C_v(K)$ would otherwise be empty then take
$C_v(K)=\{0\}$.
\end{defn}

Note that $P_v\isom\bbR^{n-1}$,
although this isometry is not especially canonical.  Differential geometers
might prefer to think of $P_v$ as the tangent space to $S^{m-1}$ at $v$.

The notation for sections may seem ambiguous, but it saves introducing another
letter, of which we are already using far too many, and it shouldn't cause
confusion, since there is no other sense in which sets are used as functions.

The definition of $C_v(K)$ really does provide enough information, because any
subset is determined by all of its sections.

It is probably best to think of this compression as chopping up $K$ into
uncountably many slices, compressing each as much as possible, and then pasting
them back together again.

The following theorem summarizes the most important properties of these
compressions that we can prove so far.

\begin{thm}
\label{thm:compressionProperties}
Let $K\in\kappa(\bbR^m)$, let $v\in S^{m-1}$ and let $C=C_v(K)$.  Then
\begin{enumerate}
\item $C$ is bounded by $\lV K\rV$.
\label{prop:compressionBounded}
\item $C$ is compact.
\label{prop:compressionCompact}
\item $\mu(C)=\mu(K)$.
\label{prop:compressionMeasurePreserving}
\item $C\in\kappa(\bbR^m)$.
\label{prop:compressionNonEmpty}
\item $\lV C\rV\leq\lV K\rV$.
\label{prop:compressionNormDecreasing}
\item Either $C=\{0\}$ or
	$\forall x\in\bbR:C(v,x)\not=\emptyset\implies K(v,x)\not=\emptyset$
\label{prop:compressionNonCreative}
\end{enumerate}
\end{thm}

\begin{proof}
All of these properties are clearly true if we are in the particular special
case mentioned in the last part of the definition, so we assume that we are not
in that case.

\ref{prop:compressionBounded}. $C$ is bounded by $\lV K\rV$:

Let $c\in C$.  We can express $c=y+xv$ with $y\in P_v$, so that $y\in C(v,x)$.

Thus, in particular, $C(v,x)$ is non-empty, and so we consider two cases.

Case 1: $C(v,x)$ is a closed ball of positive radius.

In this case
$\mu(C(v,x))>0$, and hence $\mu(K(v,x))>0$, so $K(v,x)\not=\emptyset$.

Furthermore, since
\[
\mu(C(v,x))=\mu(K(v,x))\leq\mu(B(0,\lV K(v,x)\rV))
\]
we certainly know that the radius
of $C(v,x)$ is at most $\lV K(v,x)\rV\leq\sqrt{\lV K\rV^2-x^2}$.

Thus $\lV y\rV\leq\sqrt{\lV K\rV^2-x^2}$, and hence $\lV c\rV\leq\lV K\rV$.

Case 2: $C(v,x)=\{0\}$

In this case we must have $y=0$, so $c=vx$ and $\lV c\rV=\lv x\rv$.

From the definition, we can find $w$
arbitrarily close to $x$, with $\mu(K(v,w))\not=0$, and thus
$K(v,w)\not=\emptyset$ so that $\lV K\rV\geq\lv w\rv$.

Hence, $\lV K\rV\geq\lv x\rv=\lV c\rV$, as required.

\ref{prop:compressionCompact}. $C$ is compact:

We know from \ref{prop:compressionBounded} that $C$ is bounded, so it suffices
to show that $C$ is closed.

Let $c_n$ be a Cauchy sequence in $C$.  Decompose $c_n=y_n+x_nv$ where
$\forall n:y_n\in P_v$.  Orthogonal projections are continuous, so each of
$x_n$ and $y_n$ is a Cauchy sequence.

Let $x=\lim x_n$, and suppose
\[
w\in\limsup_{n\to\infty}K(v,x_n)
\]
Then we can find a subsequence $x_{n_k}$ with
%
\begin{eqnarray*}
\forall k&:&w\in K(v,x_{n_k}) \\
\forall k&:&w+x_{n_k}v\in K
\end{eqnarray*}
%
But $w+x_{n_k}v\to w+xv$ as $n\to\infty$, and $K$ is closed, so $w+xv\in K$,
and thus $w\in K(v,x)$.

Hence, we deduce that
%
\begin{eqnarray*}
\limsup_{n\to\infty} K(v,x_n) &\subseteq& K(v,x) \\
\limsup_{n\to\infty}\mu(K(v,x_n)) &\leq& \mu(K(v,x)) \\
\limsup_{n\to\infty}\mu(C(v,x_n)) &\leq& \mu(C(v,x)) \\
\limsup_{n\to\infty}C(v,x_n) &\subseteq& C(v,x)
\end{eqnarray*}
%
where the last step is clear when $\limsup_{n\to\infty}\mu(C(v,x_n))>0$
since all the $C(v,x)$ are closed balls centred at $0$ (by considering the
sequence of radii),
but requires further explanation when $\limsup_{n\to\infty}\mu(C(v,x_n))=0$.

In this case we forsee a possible problem when $C(v,x)=\emptyset$, but if that
were the case then (by the definition)
there must be some open neighbourhood $U$ of $x$ such that
$\forall y\in U:\mu(K(v,x))=0$.  Then, eventually $x_n\in U$, and so
$x_n$ has a neigbourhood with the same property, which means
$C(v,x_n)=\emptyset$,
from which we deduce $\limsup_{n\to\infty}C(v,x_n)=\emptyset$, and the asserted
deduction does indeed still hold in this case.

Now let $y=\lim_{n\to\infty} y_n$, so $y\in P_v$,
and $\lV y\rV=\lim_{n\to\infty}\lV y_n\rV$.  Thus,
recalling that $y_n\in C(v,x_n)$, and again exploiting the fact that the
$C(v,x)$ are closed balls centred at the origin, so the radius of $C(v,x_n)$ is
at least $\lV y_n\rV$, we get $y\in\overline{\limsup_{n\to\infty}C(v,x_n)}$,
and hence $y\in C(v,x)$
(since $C(v,x)$ is closed and contains $\limsup_{n\to\infty}C(v,x_n)$).

Thus $y+xv\in C$.  But $y+xv=\lim_{n\to\infty} y_n+x_nv=\lim_{n\to\infty} c_n$,
so Cauchy sequences
converge in $C$, and hence $C$ is closed.

\ref{prop:compressionMeasurePreserving}. $\mu(C)=\mu(K)$:

Since $C$ is compact, we know that $C$ is measurable, and, writing
$1_A:\bbR^m\to\{0,1\}$ for the indicator
function of $A\subseteq\bbR^m$, we find:
%
\begin{eqnarray*}
\mu(C)
&=& \int_{\bbR^m}1_C\rmd\mu \\
&=& \int_\bbR\int_{P_v}1_{C(v,x)}\rmd\mu\rmd x \\
&& \textrm{By Fubini's theorem} \\
&=& \int_\bbR\mu(C(v,x))\rmd x \\
&=& \int_\bbR\mu(K(v,x))\rmd x \\
&=& \int_\bbR\int_{P_v}1_{K(v,x)}\rmd\mu\rmd x \\
&=& \int_{\bbR^m}1_K\rmd\mu \\
&& \textrm{By Fubini's theorem} \\
&=& \mu(K)
\end{eqnarray*}

\ref{prop:compressionNonEmpty}.
$C\in\kappa(\bbR^m)$:

This follows at once from \ref{prop:compressionCompact}, since we have ensured
that $C\not=\emptyset$.

\ref{prop:compressionNormDecreasing}.
$\lV C\rV\leq\lV K\rV$:

By \ref{prop:compressionNonEmpty} we know $\lV C\rV$ is meaningful, and then
the result follows at once from \ref{prop:compressionBounded}.

\ref{prop:compressionNonCreative}. Either $C=\{0\}$ or
	$\forall x\in\bbR:C(v,x)\not=\emptyset\implies K(v,x)\not=\emptyset$:

The first possibility is necessary only for the exceptional special case
already dismissed.  In any other situation the only possible source of a
problem is when $C(v,x)=\{0\}$.

By the definition we can find $x_n\to x$ with $\mu(K(v,x_n))>0$, and so we can
choose $z_n\in K(v,x_n)$.  By compactness of $K$ we can find a convergent
subsequence $z_{n_k}\to z$, and we must have $z\in K(v,x)$, so indeed
$K(v,x)\not=\emptyset$.
\end{proof}

\begin{corollary}
\label{thm:compressionNice}
Let $R,V\in\bbR$.  Then
\[
C_v(\kappa(\bbR^m,R,V))\subseteq\kappa(\bbR^m,R,V)
\]
\end{corollary}

This theorem provides us the first of the three requirements we had for our
compression (volume-preserving),
and property \ref{prop:compressionNormDecreasing} goes some way to
providing the third (making the set more like a closed ball),
but we can do better, as the following result makes clear:

\begin{thm}
\label{thm:compressionStrictlyNormDecreasing}
Let $m\geq2, K\in\kappa(\bbR^m)$ with $K$ not a closed ball centred at $0$.

Then we can find a sequence $v_1,v_2,\ldots,v_n$
in $S^{m-1}$ such that, taking $K_0=K$ and for $1\leq j\leq n$, setting
$K_j=C_{v_j}(K_{j-1})$, we get $\lV K_n\rV<\lV K\rV$.
\end{thm}

\begin{proof}
Let $M=\lV K\rV$ and $S=\{x\in\bbR^m:\lV x\rV=M\}$.  Since $K$ is not a closed
ball, we cannot have $K=\{0\}$, and so $M>0$.  Note that, by
\ref{thm:compressionProperties}.\ref{prop:compressionNormDecreasing},
for $L$ any iterated compression of $K$, $\lV L\rV\leq M$.

For $x\in[-M,M]$ let $M_x=\sqrt{M^2-x^2}$ so
that for any $v\in S^{m-1}$, and $L$ any iterated
compression of $K$, $\lV L(v,x)\rV\leq M_x$, but always $\lV S(v,x)\rV=M_x$.

If $S\cap K=S$ then since
$K\not=B(0,\lV K\rV)$, we can find $y\not\in K$ with $\lV y\rV<\lV K\rV$.
Applying any $v$-compression to $K$, we see that $K(v,y.v)$ omits some open
neighbourhood of $y-y.v$, which has positive measure, so
\[
\mu(K(v,y.v))<\mu(B(0,M_{y.v}))
\]
Thus $C_v(K)(v,y.v)$ has radius less than $M_{y.v}$,
so that $C_v(K)(v,x.v)\cap S(v,x.v)=\emptyset$, and hence
$C_v(K)\cap S\not=S$.

Thus, at the cost of one additional compression, we may assume that
$S\cap K\not=S$.

$S\cap K$ is closed in $S$, so we can find $w\in S$ and $\epsilon>0$ s.t.
$B(w,\epsilon)\cap S\cap K=\emptyset$.

$S$ is compact, so we can cover it with finitely many balls of radius
$\epsilon$.  Choose $\{u_i:i\in[n]\}\subseteq S$ (where $[n]=\{1,\ldots,n\}$)
such that
\[
S\subseteq\bigcup_{i=1}^n B(u_i,\epsilon)
\]
Now take
\[
v_i=\frac{u_i+w}{\lV u_i+w\rV}
\]
And define $K_i$ as in the statement of the theorem.
We show by induction on $i$
that
%
\begin{eqnarray*}
\forall i\in[n] &:& B(w,\epsilon)\cap S\cap K_i=\emptyset \\
\forall i\in[n]:\forall j\in[i] &:& B(u_j,\epsilon)\cap S\cap K_i=\emptyset
\end{eqnarray*}
%
To deduce the first statement, and the second for $j<i$ it suffices to show
that if $s\in S$ and $B(s,\epsilon)\cap S\cap K_{i-1}=\emptyset$ then
$B(s,\epsilon)\cap S\cap K_i=\emptyset$ (which we use by applying it with
$s=w,u_1,\ldots,u_{i-1}$).

To see this, take any $z\in B(s,\epsilon)\cap S$ and decompose $z=y+xv_i$ where
$y\in P_{v_i}$, so $\lV y\rV=M_x$.  Consider $K_{i-1}(v_i,x)$.
We know $y\not\in K_{i-1}(v_i,x)$, so
some open neighbourhood of $y$ is disjoint from $K_{i-1}(v_i,x)$, and so, as
before we see that
\[
\mu(K_{i-1}(v_i,x))<\mu(B(0,M_x))
\]
and deduce $K_{i}(v_i,x)=C_{v_i}(K_{i-1})(v_i,x)$ is disjoint from $S(v_i,x)$,
so in particular $y\not\in K_{i}(v_i,x)$ and thus $z\not\in K_{i}$.  But $z$
was an arbitrary element of $B(s,\epsilon)\cap S$, so
$B(s,\epsilon)\cap S\cap K_{i}=\emptyset$.

Now to show that $B(u_i,\epsilon)\cap S\cap K_{i}=\emptyset$, we note that
%
\begin{eqnarray*}
  u_i.v_i
  &=& \frac{u_i.u_i+u_i.w}{\lV u_i+w\rV} \\
  &=& \frac{1+u_i.w}{\lV u_i+w\rV} \\
  &=& \frac{w.w+u_i.w}{\lV u_i+w\rV} \\
  &=& w.v_i
\end{eqnarray*}
%
so $B(u_i,\epsilon)$ intersects all the same $v_i$-sections as
$B(w,\epsilon)$, and in the same way as both the arguments above,
all of these sections, after
compression, are disjoint from the same sections of $S$, and hence,
in particular,
$B(u_i,\epsilon)\cap S\cap K_{i}=\emptyset$.

In conclusion, since $S\subseteq\bigcup_{i=1}^n B(u_i,\epsilon)$, we find
that $K_n\cap S=\emptyset$, and thus $\lV K_n\rV<M$.
\end{proof}

Note that for $m>2$ this argument is extremely overcomplicated and in fact just
two compressions will always suffice, but when $m=2$ we really can need
arbitrarily many compressions to achieve a strict reduction of $\lV K\rV$.

This theorem explains precisely what we mean by ``making the set more like a
closed ball'', and shows that we can achieve this goal.

So, now it remains only to show the the compressions do not increase surface
area, and to tie up all the loose ends.

%%%%%%%%%%%%%%%%%%%%%%%%%%%%%%%%%%%%%%%%%%%%%%%%%%%%%%%%%%%%%%
%%%%%%%%%%%%%%%%%%%%%%%%%%%%%%%%%%%%%%%%%%%%%%%%%%%%%%%%%%%%%%
%%%%%%%%%%%%%%%%%%%%%%%%%%%%%%%%%%%%%%%%%%%%%%%%%%%%%%%%%%%%%%
\section{The isoperimetric inequality}
We are nearly ready to state and prove the basic form of the
isoperimetric inequality for compact subsets.  This will be an induction on
$m$, so first we must do a special case proof for $m=1$.

\begin{lemma}
Let $K\subseteq\bbR$ be compact.  Then either $K=\emptyset$ so $K$ is a closed
ball, or else $S_\epsilon(K)-\mu(K)\geq 2\epsilon$,
whereas for any non-empty closed
ball $B$, we have $S_\epsilon(B)-\mu(B)=2\epsilon$.
\end{lemma}

\begin{proof}
The case $K=\emptyset$ is clear.

For the next part, note that:
%
\begin{eqnarray*}
N_\epsilon(K)\setminus K &\supseteq&
	[\inf K-\epsilon,\inf K)\cup(\sup K,\sup K+\epsilon] \\
\mu(N_\epsilon(K)\setminus K)&\geq&
	\mu([\inf K-\epsilon,\inf K)\cup(\sup K,\sup K+\epsilon]) \\
\mu(N_\epsilon(K)\setminus K)&\geq& 2\epsilon \\
S_\epsilon(K)-\mu(K)&\geq& 2\epsilon
\end{eqnarray*}
%
And finally non-empty closed balls are exactly sets of the form $[a,b]$ with
$b\geq a$, and
clearly $N_\epsilon([a,b])=[a-\epsilon,b+\epsilon]$, and hence
$S_\epsilon([a,b])=b-a+2\epsilon$, so $S_\epsilon(B)-\mu(B)=2\epsilon$ as
required.
\end{proof}

\begin{corollary}
\label{thm:mainTheoremForR}
Let $K\in\kappa(\bbR)$, let $B\subseteq \bbR$ be the non-empty closed
ball centred at $0$ with $\mu(B)=\mu(K)$ and let $\epsilon\geq0$, then
$S_\epsilon(K)\geq S_\epsilon(B)$.
\end{corollary}

So, the main theorem is

\begin{thm}
Let $K\in\kappa(\bbR^m)$, let $B\subseteq \bbR^m$ be the non-empty closed
ball centred at $0$ with $\mu(B)=\mu(K)$ and let $\epsilon\geq0$, then
$S_\epsilon(K)\geq S_\epsilon(B)$.
\end{thm}

\begin{proof}
We work by induction on $m$.  The case $m=1$ is precisely
\ref{thm:mainTheoremForR}, so we must perform the inductive step for $m\geq2$.

For this we need the following sub-lemma, which provides the last of the
requirements we wished for regarding our compressions.

\begin{lemma}
\label{thm:compressionSurfaceAreaDecreasing}
Let $K\in\kappa(\bbR^m)$ and let $v\in S^{m-1}$.  Then
$S_\epsilon(C_v(K))\leq S_\epsilon(K)$.
\end{lemma}

\begin{proof}
Let $C=C_v(K)$.

If $C=\{0\}$ then, since $K\not=\emptyset$, the result is obvious, and so we
dismiss this case.

$P_v\isom\bbR^{m-1}$, so we can apply the inductive hypothesis to each
non-empty section $K(v,x)$ to deduce that
%
\begin{eqnarray*}
\forall x:\forall\eta\geq0 &:& S_\eta(C(v,x))\leq S_\eta(K(v,x)) \\
\forall x:\forall\eta\geq0 &:& \mu(N_\eta(C(v,x)))\leq \mu(N_\eta(K(v,x)))
\end{eqnarray*}
%
Furthermore, we note that, by rewriting distances in terms of distances
parallel and perpendicular to $v$, we get
%
\begin{eqnarray*}
N_\epsilon(K)(v,x) &=&
	\bigcup_{z\in[-\epsilon,\epsilon],K(v,x+z)\not=\emptyset}
	N_{\sqrt{\epsilon^2-z^2}}(K(v,x+z)) \\
N_\epsilon(C)(v,x) &=&
	\bigcup_{z\in[-\epsilon,\epsilon],C(v,x+z)\not=\emptyset}
	N_{\sqrt{\epsilon^2-z^2}}(C(v,x+z))
\end{eqnarray*}
%
If all the $C(v,x+z)$ are empty, then we are done, but otherwise we proceed as
follows:

Now each non-empty $C(v,x+z)$ is a non-empty closed
ball centred at $0$, and thus the corresponding
$N_{\sqrt{\epsilon^2-z^2}}(C(v,x+z))$ are also
non-empty closed balls centred at $0$.

So, all the $N_{\sqrt{\epsilon^2-z^2}}(C(v,x+z))$ are nested.  Let $R$ be the
supremal radius of these balls, and find a sequence
$z_n$ such that the radius of $N_{\sqrt{\epsilon^2-z_n^2}}(C(v,x+z_n))$ tends
to $R$ as $n\to\infty$, and each $N_{\sqrt{\epsilon^2-z_n^2}}(C(v,x+z_n))$ is
non-empty.

By taking a convergent subsequence of $z_n$ and applying the compactness of
$C$, we can find $z_0$ such that
\[
\forall z\in[-\epsilon,\epsilon]:N_{\sqrt{\epsilon^2-z^2}}(C(v,x+z))\subseteq
	N_{\sqrt{\epsilon^2-z_0^2}}(C(v,x+z_0))
\]
and hence
\[
N_\epsilon(C)(v,x)=N_{\sqrt{\epsilon^2-z_0^2}}(C(v,x+z_0))
\]
so in particular this set is non-empty, and by
\ref{thm:compressionProperties}.\ref{prop:compressionNonCreative}, $K(v,x+z_0)$
is also non-empty.

Now we see that
%
\begin{eqnarray*}
\mu(N_\epsilon(C)(v,x))
&=& \mu\left(N_{\sqrt{\epsilon^2-z_0^2}}(C(v,x+z_0))\right) \\
&\leq& \mu\left(N_{\sqrt{\epsilon^2-z_0^2}}(K(v,x+z_0))\right) \\
&& \textrm{By inductive hypothesis} \\
&\leq& \mu\left(\bigcup_{z\in[-\epsilon,\epsilon]}
	N_{\sqrt{\epsilon^2-z^2}}(K(v,x+z))\right) \\
&\leq& \mu(N_\epsilon(K)(v,x))
\end{eqnarray*}
%
Hence, by integrating over $x$ and applying Fubini's theorem we deduce
\[
\mu(N_\epsilon(C))\leq\mu(N_\epsilon(K))
\]
as required.
\end{proof}

Now we finish the proof of the main theorem by tying together all of the
results we have proved.

If $\epsilon=0$ then $S_\epsilon(K)=\mu(K)=\mu(B)=S_\epsilon(B)$, so the result
is true, and henceforth we assume $\epsilon>0$.

Let $\kappa=\kappa(\bbR^n,\lV K\rV,\mu(K))$.

By \ref{thm:volumeBoundedSpaces}, $\kappa$ is compact and by
\ref{thm:surfaceAreaContinuous}, $S_\epsilon:\kappa\to\bbR$ is continuous.
Thus $S_\epsilon$ achieves its minimum $A$.

Let $X=S_\epsilon^{-1}(\{A\})$.  $X$ is closed in $\kappa$, so $X$ is compact.

$\lV\cdot\rV:X\to\bbR$ is also continuous, and also achieves a minimum value
$R$.  Let $C$ be a set at which this minimum is achieved.

Suppose $C$ is not a closed ball centred at $0$, then we can apply
\ref{thm:compressionStrictlyNormDecreasing} to get $C'$, an iterated
compression of $C$ with $\lV C'\rV<R$.

By \ref{thm:compressionNice} we know that $C'\in\kappa$.
By \ref{thm:compressionSurfaceAreaDecreasing} we also know that
$S_\epsilon(C')\leq S_\epsilon(C)$ so $C'\in X$, but this contradicts the
choice of $C$.

Hence, it must be that $C$ is in fact a non-empty closed ball centred at $0$.
Take $C=B(0,R)$.

Now consider $B$ (recall that we defined $B$ to the the closed ball at $0$ of
volume $\mu(K)$).  Let $r$ be the radius of $B$.  Certainly $\mu(B)\leq\mu(C)$
(since $C\in\kappa=\kappa(\bbR^m,\lV K\rV,\mu(B))$), and thus
$r\leq R$.  Hence $S_\epsilon(B)\leq S_\epsilon(C)$ and since $C\in X$ we also
know $S_\epsilon(C)\leq S_\epsilon(K)$, and indeed we have
$S_\epsilon(B)\leq S_\epsilon(K)$ as required.

(In fact, of course, $B=C$, but we don't need to prove that)
\end{proof}

\begin{corollary}
[Isoperimetric inequality for compact subsets of Euclidean space]
Amongst non-empty compact subsets of $\bbR^m$ of any fixed volume, surface area
is minimised by the non-empty closed balls of that volume.
\end{corollary}

\begin{proof}
Take $K$ and $B$ as in the theorem above.

For each fixed $\epsilon>0$, we have $S_\epsilon(K)\geq S_\epsilon(B)$.  Also
$\mu(K)=\mu(B)$, so when we subtract this, divide by $\epsilon$
and let $\epsilon\to0^+$ we certainly
get $S(B)\leq S(K)$.
\end{proof}

\begin{comment}
\begin{corollary}
Combining this with \ref{thm:smoothSurfaceAreas}, we deduce that amongst all
smooth compact submainfolds (with boundary) of $\bbR^m$ of dimension $m$ of any
fixed volume, closed balls minimise surface area.
\end{corollary}
\end{comment}

%%%%%%%%%%%%%%%%%%%%%%%%%%%%%%%%%%%%%%%%%%%%%%%%%%%%%%%%%%%%%%
%%%%%%%%%%%%%%%%%%%%%%%%%%%%%%%%%%%%%%%%%%%%%%%%%%%%%%%%%%%%%%
%%%%%%%%%%%%%%%%%%%%%%%%%%%%%%%%%%%%%%%%%%%%%%%%%%%%%%%%%%%%%%
\section{Generalizations}
There are at least two obvious directions to generalize this result.

We might try to generalize it to other norms on $\bbR^m$.  In this case the
difficulty will be proving the analogue of
\ref{thm:compressionStrictlyNormDecreasing} in two dimensions - we were only
able to do this because the group of norm-preseving linear maps (i.e. the
orthogonal group) acts transitively on the unit sphere for the euclidean
metric.  This would not be true for most other norms.  Probably the case $m=2$
will become another special case which will have to be handled more carefully.

Another possibility is to generalize to compact subsets of some other
Riemannian manifold.  In this case one must first be careful to be sure one
understands exactly what the analogue of volume is, and even then there is no
obvious analogue of our compressions, about which the entire proof hinged.
\end{document}
